\documentclass[12pt]{article}
\usepackage[utf8]{inputenc}
\usepackage[spanish]{babel}
\usepackage{amsmath, amssymb}
\usepackage{graphicx}

\title{Tema 20: Circuitos Eléctricos \\[0.3cm]
	Parte B – Análisis de Bifurcación}
\author{Alejandro López Castro}
\date{Curso 2025–2026}

\begin{document}
	
	\maketitle
	
	\section*{Introducción}
	
	En esta parte se analiza el comportamiento no lineal de un circuito eléctrico que contiene un diodo, elemento cuya relación corriente–voltaje no es lineal. Este tipo de circuitos puede presentar \textbf{bifurcaciones}, es decir, cambios cualitativos en su dinámica cuando se modifica el voltaje aplicado. 
	
	El modelo simplificado que representa este fenómeno se describe mediante la ecuación diferencial:
	
	\begin{equation}
		\frac{dz}{dt} = \mu - z^2,
	\end{equation}
	
	donde $z(t)$ representa la corriente y $\mu$ es un parámetro de control asociado al voltaje aplicado. Este modelo muestra cómo el comportamiento del circuito cambia según el valor de $\mu$.
	
	\section*{1. Determinación de los puntos de equilibrio}
	
	Los puntos de equilibrio se obtienen imponiendo la condición de equilibrio:
	
	\[
	\frac{dz}{dt} = 0.
	\]
	
	De la ecuación se deduce:
	\begin{equation}
		\mu - z^2 = 0,
	\end{equation}
	
	y por tanto:
	\[
	z = \pm \sqrt{\mu}.
	\]
	
	Esto significa que:
	\begin{itemize}
		\item Si $\mu > 0$, existen dos equilibrios reales: $z_1 = +\sqrt{\mu}$ y $z_2 = -\sqrt{\mu}$.
		\item Si $\mu = 0$, ambos colapsan en un único punto $z = 0$.
		\item Si $\mu < 0$, no existen equilibrios reales.
	\end{itemize}
	
	\section*{2. Estabilidad de los equilibrios}
	
	Para estudiar la estabilidad se linealiza la ecuación alrededor de un equilibrio $z^*$:
	
	\[
	\frac{d(\delta z)}{dt} = \left.\frac{d}{dz}(\mu - z^2)\right|_{z = z^*} \delta z.
	\]
	
	Como 
	\[
	\frac{d}{dz}(\mu - z^2) = -2z,
	\]
	entonces la ecuación linealizada queda:
	\[
	\frac{d(\delta z)}{dt} = -2z^* \, \delta z.
	\]
	
	De aquí:
	\begin{itemize}
		\item Si $z^* = +\sqrt{\mu}$, el coeficiente $-2\sqrt{\mu} < 0$ $\Rightarrow$ equilibrio \textbf{estable}.
		\item Si $z^* = -\sqrt{\mu}$, el coeficiente $+2\sqrt{\mu} > 0$ $\Rightarrow$ equilibrio \textbf{inestable}.
	\end{itemize}
	
	Por lo tanto, cuando $\mu > 0$, el sistema posee un equilibrio estable y otro inestable.
	
	\section*{3. Diagrama de bifurcación}
	
	El diagrama se representa en el plano $(\mu, z)$:
	
	\begin{itemize}
		\item Para $\mu < 0$, no hay equilibrios reales.
		\item En $\mu = 0$, ambos equilibrios coinciden en el origen.
		\item Para $\mu > 0$, aparecen dos ramas: $z = +\sqrt{\mu}$ (estable) y $z = -\sqrt{\mu}$ (inestable).
	\end{itemize}
	
	Esto corresponde a una \textbf{bifurcación tipo silla–nodo}, donde dos equilibrios surgen a partir de un valor crítico del parámetro.
	
	\begin{center}
		\includegraphics[width=0.65\textwidth]{bifurcacion.png}
	\end{center}
	
	\textbf{Figura:} Diagrama de bifurcación del modelo $\dot{z} = \mu - z^2$. La rama superior es estable y la inferior inestable.
	
	\section*{4. Interpretación física}
	
	En el contexto del circuito:
	\begin{itemize}
		\item $\mu$ representa el \textbf{voltaje aplicado}.
		\item $z(t)$ representa la \textbf{corriente} que atraviesa el diodo.
	\end{itemize}
	
	Cuando $\mu < 0$, el voltaje no alcanza el umbral necesario y el sistema no tiene un estado estacionario real: el diodo permanece “apagado”, sin conducción.  
	Cuando $\mu > 0$, aparecen dos posibles estados estacionarios:
	\begin{itemize}
		\item Uno estable, que corresponde a la corriente positiva (conducción normal del diodo).
		\item Otro inestable, que representaría una corriente negativa físicamente irreal.
	\end{itemize}
	
	Así, el cambio de signo de $\mu$ simboliza el paso de un régimen sin corriente a un régimen estable de conducción. Este comportamiento es análogo a la activación de un diodo real al superar su voltaje umbral.
	
	\section*{5. Conclusión}
	
	El modelo $\dot{z} = \mu - z^2$ muestra cómo un circuito eléctrico no lineal puede cambiar su comportamiento de forma cualitativa al variar el voltaje aplicado.  
	Cuando $\mu$ cruza el valor crítico $0$, el sistema pasa de no tener equilibrios (sin conducción) a tener dos (uno estable y otro inestable), lo que describe perfectamente la aparición de un estado de conducción estable en el circuito con diodo.
	
	Este análisis evidencia cómo las bifurcaciones permiten comprender la transición entre diferentes regímenes dinámicos en sistemas eléctricos no lineales.
	
\end{document}
