\documentclass{article}
\usepackage[utf8]{inputenc}
\usepackage[spanish]{babel}
\usepackage{amsmath}
\usepackage{amssymb}
\usepackage{booktabs} % Para tablas más profesionales
\usepackage{caption} % Para mejor control de captions
\usepackage{float} % <-- AÑADE ESTE PAQUETE

\title{Solución del Circuito RC}
\author{}
\date{}

\begin{document}

\maketitle

\section*{Parte A}

\subsection*{a. Ecuación diferencial para la carga \( Q(t) \)}

La ecuación del circuito es:
\[ R \cdot I(t) + \frac{1}{C}Q = E(t) \]
donde \( E(t) = E_0 \) (constante) y \( I(t) = Q'(t) \).

Sustituyendo:
\[ R \frac{dQ}{dt} + \frac{1}{C}Q = E_0 \]

Esta es una ecuación diferencial de primer orden:
\[ \frac{dQ}{dt} + \frac{1}{RC}Q = \frac{E_0}{R} \]

El factor integrante es:
\[ \mu(t) = e^{\int \frac{1}{RC} dt} = e^{\frac{t}{RC}} \]

Multiplicando la ecuación por el factor integrante:
\[ e^{\frac{t}{RC}} \frac{dQ}{dt} + \frac{1}{RC} e^{\frac{t}{RC}} Q = \frac{E_0}{R} e^{\frac{t}{RC}} \]

Reconociendo la derivada del producto:
\[ \frac{d}{dt} \left( Q e^{\frac{t}{RC}} \right) = \frac{E_0}{R} e^{\frac{t}{RC}} \]

Integrando ambos lados:
\[ Q e^{\frac{t}{RC}} = \frac{E_0}{R} \int e^{\frac{t}{RC}} dt = E_0 C e^{\frac{t}{RC}} + K \]

Despejando \( Q(t) \):
\[ Q(t) = E_0 C + K e^{-\frac{t}{RC}} \]

Usando la condición inicial \( Q(0) = 0 \):
\[ 0 = E_0 C + K \Rightarrow K = -E_0 C \]

Por lo tanto:
\[ Q(t) = E_0 C (1 - e^{-\frac{t}{RC}}) \]

La corriente es:
\[ I(t) = \frac{dQ}{dt} = \frac{E_0}{R} e^{-\frac{t}{RC}} \]

\subsection*{b. Verificar que límite \( Q(t) = E_0 C \), límite \( I(t) = 0 \)}

Para la carga:
\[ \lim_{t \to \infty} Q(t) = \lim_{t \to \infty} E_0 C (1 - e^{-\frac{t}{RC}}) = E_0 C (1 - 0) = E_0 C \]

Para la corriente:
\[ \lim_{t \to \infty} I(t) = \lim_{t \to \infty} \frac{E_0}{R} e^{-\frac{t}{RC}} = \frac{E_0}{R} \cdot 0 = 0 \]

\section*{Comparación de los métodos numéricos con la solución analítica}
Para comparar la exactitud de los métodos utilizaremos un ejemplo con los parámetros de: \[E_0 = 110, R = 50, C = 0.05, h = 0.1\] Además usaremos como valor real t = 10 y valor aproximado t = 9,9

% Cambia todas las tablas a [H]
\begin{table}[H]
\centering
\begin{tabular}{cc}
\toprule
\textbf{Tiempo (t)} & \textbf{Valor} \\
\midrule
9,9 & 5,399264 \\
10 & 5,395153 \\
\bottomrule
\end{tabular}
\quad
\begin{minipage}{0.5\textwidth}
Conociendo ambos de estos resultados se concluye que el error relativo hacia atrás es de 0.000761
\end{minipage}
\caption{Solución analítica con el valor aproximado y el valor real}
\end{table}

\begin{table}[H]
\centering
\begin{tabular}{cc}
\toprule
\textbf{Tiempo (t)} & \textbf{Valor} \\
\midrule
9,7 & 5,395125 \\
9,8 & 5,399320 \\
9,9 & 5,403347 \\
\bottomrule
\end{tabular}
\quad
\begin{minipage}{0.5\textwidth}
Dando un error relativo hacia adelante de un 0.001517 y un número de condición de 1.993429
\end{minipage}
\caption{Método de Euler}
\end{table}

\begin{table}[H]
\centering
\begin{tabular}{cc}
\toprule
\textbf{Tiempo (t)} & \textbf{Valor} \\
\midrule
9,7 & 5,386299 \\
9,8 & 5,390756 \\
9,9 & 5,395039 \\
\bottomrule
\end{tabular}
\quad
\begin{minipage}{0.5\textwidth}
Dando un error relativo hacia adelante de un 0.000021 y un número de condición de 0.027595
\end{minipage}
\caption{Método Euler Mejorado}
\end{table}

\begin{table}[H]
\centering
\begin{tabular}{cc}
\toprule
\textbf{Tiempo (t)} & \textbf{Valor} \\
\midrule
9,7 & 5,386420 \\
9,8 & 5,390874 \\
9,9 & 5,395153 \\
\bottomrule
\end{tabular}
\quad
\begin{minipage}{0.5\textwidth}
Dando un error relativo hacia adelante de un 0.000761 y un número de condición de 1
\end{minipage}
\caption{Método de Runge - Kutta 4}
\end{table}

\noindent
Con el objetivo de calcular el orden de convergencia de dichos métodos haremos 10 iteraciones de ellos por cada uno de los siguientes valores de h = { 0.5; 0.2; 0.1; 0.05; 0.02; 0.01 } para asi obtener el orden entre cada uno de los tamaños de pasos usando la siguiente fórmula:

\vspace{0.5cm}

\noindent
\textbf{Fórmula del orden de convergencia:} 
\[
p = \frac{\log\left(\frac{\text{Error}_{h_1}}{\text{Error}_{h_2}}\right)}{\log\left(\frac{h_1}{h_2}\right)}
\]

\vspace{0.5cm}

% CAMBIA ESTA TABLA TAMBIÉN A [H]
\begin{table}[H]
\centering
\caption{Órdenes de convergencia de los métodos numéricos}
\begin{tabular}{|c|c|c|c|c|c|c|}
\hline
\textbf{Método} & \textbf{0.5→0.2} & \textbf{0.2→0.1} & \textbf{0.1→0.05} & \textbf{0.05→0.02} & \textbf{0.02→0.01} & \textbf{Promedio} \\
\hline
Euler & 1.0595 & 1.0249 & 1.0122 & 1.0055 & 1.0024 & 1.0209 \\
\hline
Euler Mejorado & 2.1025 & 2.0429 & 2.0218 & 2.0099 & 2.0043 & 2.0363 \\
\hline
Runge-Kutta 4 & 4.1104 & 4.0470 & 4.0241 & 4.0109 & 4.0044 & 4.0394 \\
\hline
\end{tabular}
\end{table}

\end{document}