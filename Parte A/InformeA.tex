\documentclass{article}
\usepackage[utf8]{inputenc}
\usepackage[spanish]{babel}
\usepackage{amsmath}
\usepackage{amssymb}

\title{Solución del Circuito RC}
\author{}
\date{}

\begin{document}

\maketitle

\section*{Parte A}

\subsection*{a. Ecuación diferencial para la carga \( Q(t) \)}

La ecuación del circuito es:
\[ R \cdot I(t) + \frac{1}{C}Q = E(t) \]
donde \( E(t) = E_0 \) (constante) y \( I(t) = Q'(t) \).

Sustituyendo:
\[ R \frac{dQ}{dt} + \frac{1}{C}Q = E_0 \]

Esta es una ecuación diferencial de primer orden:
\[ \frac{dQ}{dt} + \frac{1}{RC}Q = \frac{E_0}{R} \]

El factor integrante es:
\[ \mu(t) = e^{\int \frac{1}{RC} dt} = e^{\frac{t}{RC}} \]

Multiplicando la ecuación por el factor integrante:
\[ e^{\frac{t}{RC}} \frac{dQ}{dt} + \frac{1}{RC} e^{\frac{t}{RC}} Q = \frac{E_0}{R} e^{\frac{t}{RC}} \]

Reconociendo la derivada del producto:
\[ \frac{d}{dt} \left( Q e^{\frac{t}{RC}} \right) = \frac{E_0}{R} e^{\frac{t}{RC}} \]

Integrando ambos lados:
\[ Q e^{\frac{t}{RC}} = \frac{E_0}{R} \int e^{\frac{t}{RC}} dt = E_0 C e^{\frac{t}{RC}} + K \]

Despejando \( Q(t) \):
\[ Q(t) = E_0 C + K e^{-\frac{t}{RC}} \]

Usando la condición inicial \( Q(0) = 0 \):
\[ 0 = E_0 C + K \Rightarrow K = -E_0 C \]

Por lo tanto:
\[ Q(t) = E_0 C (1 - e^{-\frac{t}{RC}}) \]

La corriente es:
\[ I(t) = \frac{dQ}{dt} = \frac{E_0}{R} e^{-\frac{t}{RC}} \]

\subsection*{b. Verificar que límite \( Q(t) = E_0 C \), límite \( I(t) = 0 \)}

Para la carga:
\[ \lim_{t \to \infty} Q(t) = \lim_{t \to \infty} E_0 C (1 - e^{-\frac{t}{RC}}) = E_0 C (1 - 0) = E_0 C \]

Para la corriente:
\[ \lim_{t \to \infty} I(t) = \lim_{t \to \infty} \frac{E_0}{R} e^{-\frac{t}{RC}} = \frac{E_0}{R} \cdot 0 = 0 \]

\end{document}