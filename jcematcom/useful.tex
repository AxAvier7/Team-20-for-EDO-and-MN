%-----------------------------------------------------------------------------------
% Para producir listas enumeradas, utilice el siguiente estilo:
% \begin{enumerate}
% 	\item Primer Elemento
% 	\item Segundo Elemento
% 	%
% 	\begin {enumerate}
% 	\item {Segundo Elemento - Subítem Uno}
% 	\item {Segundo Elemento - Subítem Dos}
% 	\end {enumerate}
% 	%
% \end{enumerate}

%-----------------------------------------------------------------------------------
% Para producir descripciones, use el siguiente estilo:

% %-----------------------------------------------------------------------------------
% \begin{description}
% 	\item [Primer Elemento] con su respectiva descripción.
% 	\item [Segundo Elemento] también con su respectiva descripción.
% \end{description}

% %-----------------------------------------------------------------------------------
% Para producir cuerpos flotantes (figuras o tablas), asegúrese de numerar
% y etiquetar correctamente cada figura. Las referencias a las figuras deben
% estar correctamente etiquetadas. Por ejemplo, véase la Fig. \ref{fig:ex}\ldots

% \begin{figure}[h!]%
% 	\begin{center}
% 		\begin{tabular}{|c|c|c|} \hline
% 			& Método 1 	& Método 2 	\\ \hline
% 			A 			&  			&  			\\ \hline
% 			B			& 			& 			\\ \hline
% 			C 			& 			&  			\\ \hline
% 		\end{tabular}
% 		\caption{Figura de ejemplo. Recuerde especificar el origen de los datos que se muestran. \label{fig:ex}}
% 	\end{center}
% \end{figure}
% %-----------------------------------------------------------------------------------

% Para producir código fuente, envuélvalo en una figura flotante y
% etiquételo correctamente. Por ejemplo, en la Fig. \ref{fig:code}
% se muestra un código bastante conocido\ldots

% % Configuración de Listings
% \lstset{keywordstyle=\color{blue}, basicstyle=\small}

% \begin{figure}[h!]%
% 	\begin{lstlisting}[language=c]%
		
%      int main(int argc, char** argv)
%      {
%          // Imprimiendo "Hola Mundo".
%          printf("Hello, World");
%      }
		
% 	\end{lstlisting}
% 	\caption{Código fuente de ejemplo.\label{fig:code}}
% \end{figure}
