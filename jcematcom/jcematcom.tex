%===================================================================================
% PREÁMBULO
%-----------------------------------------------------------------------------------
\documentclass[a4paper,10pt,twocolumn]{article}

%===================================================================================
% Paquetes
%-----------------------------------------------------------------------------------
\usepackage{amsmath}
\usepackage{amsfonts}
\usepackage{amssymb}
\usepackage{jcematcom}
\usepackage[utf8]{inputenc}
\usepackage{listings}
\usepackage[pdftex]{hyperref}
\usepackage{caption}
\usepackage{subcaption}
\usepackage{float}
\usepackage{booktabs}
%-----------------------------------------------------------------------------------
% Configuración
%-----------------------------------------------------------------------------------
\hypersetup{colorlinks,%
	    citecolor=black,%
	    filecolor=black,%
	    linkcolor=black,%
	    urlcolor=blue}

%===================================================================================



%===================================================================================
% Presentacion
%-----------------------------------------------------------------------------------
% Título
%-----------------------------------------------------------------------------------
\title{Proyecto Conjunto de Ecuaciones Diferenciales Ordinarias y Matemática Numérica, Equipo No. 20}

%-----------------------------------------------------------------------------------
% Autores
%-----------------------------------------------------------------------------------
\author{\\
\name Brioso Jurado, Adrián Xavier \email \href{adrian.xbrioso@estudiantes.matcom.uh.cu}{adrian.xbrioso@estudiantes.matcom.uh.cu}
	\\ \addr Grupo B211 \AND
\name García Ruiz, Alejandro \email \href{alejandro.garcia@estudiantes.matcom.uh.cu}{alejandro.garcia@estudiantes.matcom.uh.cu}
  \\ \addr Grupo B212 \AND
\name López Castro, Alejandro \email \href{alejandro.lopez@estudiantes.matcom.uh.cu}{alejandro.lopez@estudiantes.matcom.uh.cu}
  \\ \addr Grupo B212}

%-----------------------------------------------------------------------------------
% Headings
%-----------------------------------------------------------------------------------
\jcematcomheading{\the\year}{1-\pageref{end}}{Brioso, Adrián Xavier; García, Alejandro; López, Alejandro}

%-----------------------------------------------------------------------------------
\ShortHeadings{Proyecto Conjunto EDO \& MN, Equipo No. 20}{Autores}
	\tutors{Ania Mesa, Damián Valdés, Jackson Vera, Amanda Noris, Juan Miguel Pérez}
%===================================================================================



%===================================================================================
% DOCUMENTO
%-----------------------------------------------------------------------------------
\begin{document}

%-----------------------------------------------------------------------------------
% NO BORRAR ESTA LINEA!
%-----------------------------------------------------------------------------------
\twocolumn[
%-----------------------------------------------------------------------------------

\maketitle

%===================================================================================
% Resumen y Abstract
%-----------------------------------------------------------------------------------
\selectlanguage{spanish}

%-----------------------------------------------------------------------------------
% Resumen en Español
%-----------------------------------------------------------------------------------
\begin{abstract}

	En este proyecto se aborda el tema de cómo las ecuaciones diferenciales ordinarias describen el comportamiento dinámico de distintos tipos de circuitos eléctricos. Se inicia presentando el circuito RC, conformado por un resistor y un capacitor, donde la ecuación diferencial
	\[
	R\,\frac{dQ}{dt} + \frac{1}{C}Q(t) = E_0
	\]
	modela la carga del capacitor tras cerrar el interruptor. Se determina la función de carga \( Q(t) \), la corriente \( I(t) \) y el comportamiento asintótico del sistema, realizando su análisis mediante un campo de isoclinas.

	Más adelante, en la parte de bifurcación, se introduce un modelo no lineal para la corriente \( z(t) \) con la EDO
	\[
	\dot z = \mu - z^{2},
	\]
	que permite estudiar cómo el voltaje \( \mu \) puede generar o eliminar estados estacionarios, a la vez que se interpreta la respuesta del circuito ante distintos regímenes de operación.

	Por último, se analiza un circuito RLC, donde la interacción entre resistencia, inductor y capacitor lleva a un sistema lineal de dos ecuaciones que describe la evolución de la carga y la corriente. Se estudia el punto crítico del sistema, su estabilidad y se interpreta el plano de fase como representación de la descarga oscilatoria del capacitor dentro del circuito.
	Para auxiliar al lector, se adjunta con este informe un archivo .ipynb que contiene explicaciones más resumidas y códigos en Python a los que se hacen referencia.

\end{abstract}

%-----------------------------------------------------------------------------------
% English Abstract
%-----------------------------------------------------------------------------------
\vspace{0.5cm}

\begin{enabstract}

	This project addresses how ordinary differential equations describe the dynamic behavior of different types of electrical circuits. It begins by presenting the RC circuit, composed of a resistor and a capacitor, where the differential equation
	\[
	R\,\frac{dQ}{dt} + \frac{1}{C}Q(t) = E_0
	\]
	models the charge of the capacitor after the switch is closed. The charge function \( Q(t) \), the current \( I(t) \), and the asymptotic behavior of the system are determined, and their analysis is carried out using an isocline field.

	Later, in the bifurcation section, a nonlinear model for the current \( z(t) \) is introduced, governed by the ODE
	\[
	\dot z = \mu - z^{2},
	\]
	which makes it possible to study how the voltage \( \mu \) can generate or eliminate steady states, while also interpreting the circuit's response under different operating regimes.

	Finally, an RLC circuit is analyzed, where the interaction between resistance, inductor, and capacitor leads to a linear system of two equations that describes the evolution of the charge and the current. The critical point of the system and its stability are examined, and the phase plane is interpreted as a representation of the oscillatory discharge of the capacitor within the circuit.
	To assist the reader, an .ipynb file is attached to this report containing more concise explanations and Python codes that are referenced throughout the document.

\end{enabstract}

%-----------------------------------------------------------------------------------
% Palabras clave
%-----------------------------------------------------------------------------------
\begin{keywords}
	EDO,
	Matemática Numérica,
	Circuito RC.
\end{keywords}

%-----------------------------------------------------------------------------------
% Temas
%-----------------------------------------------------------------------------------
\begin{topics}
	Ecuaciones Diferenciales Ordinarias y Matemática Numérica aplicada a un circuito eléctrico.
\end{topics}


%-----------------------------------------------------------------------------------
% NO BORRAR ESTAS LINEAS!
%-----------------------------------------------------------------------------------
\vspace{0.8cm}
]
%-----------------------------------------------------------------------------------


%===================================================================================
% Resumen Extendido
%-----------------------------------------------------------------------------------
\section{Resumen Extendido}\label{sec:intro}
%-----------------------------------------------------------------------------------
\subsection{Modelado Matemático del Circuito RLC}

Un circuito RLC en serie consta de tres componentes fundamentales: resistencia ($R$), inductancia ($L$) y capacitancia ($C$), conectados en serie con una fuente de voltaje $E(t)$. Las variables clave son la carga $Q(t)$ en el capacitor y la corriente $I(t)$, relacionadas por:

\[
\frac{dQ}{dt} = I(t)
\]

Aplicando la ley de Kirchhoff, se obtiene la ecuación del circuito:

\[
L\frac{dI}{dt} + RI + \frac{1}{C}Q = E(t)
\]

Derivando y sustituyendo, se llega a la EDO de segundo orden para la corriente:

\[
LI'' + RI' + \frac{1}{C}I = E'(t)
\]

\subsection{Analogía Mecánica-Eléctrica}

Existe una correspondencia directa entre el circuito RLC y los sistemas masa-resorte-amortiguador:

\begin{center}
\begin{tabular}{|c|c|}
\hline
\textbf{Sistema Mecánico} & \textbf{Sistema Eléctrico} \\
\hline
Masa $m$ & Inductancia $L$ \\
Amortiguamiento $c$ & Resistencia $R$ \\
Constante del resorte $k$ & Recíproco de capacitancia $1/C$ \\
Posición $x$ & Carga $Q$ o corriente $I$ \\
Fuerza $F(t)$ & Voltaje $E(t)$ o $E'(t)$ \\
\hline
\end{tabular}
\end{center}

Esta analogía permite trasladar resultados entre dominios físicos diferentes.

\subsection{Solución para Corriente Alterna}

Para un voltaje de entrada $E(t) = E_0 \sin \omega t$, la solución completa es:

\[
I(t) = I_{tr}(t) + I_{sp}(t)
\]

donde $I_{tr}(t)$ es la corriente transitoria (decae exponencialmente) e $I_{sp}(t)$ es la corriente estacionaria periódica.

\subsection{Impedancia y Fase}

La \textbf{impedancia} del circuito se define como:

\[
Z = \sqrt{R^2 + (\omega L - \frac{1}{\omega C})^2}
\]

La corriente estacionaria tiene amplitud $I_0 = E_0/Z$ y presenta un \textbf{retraso de fase} $\delta$:

\[
I_{sp}(t) = \frac{E_0}{Z} \sin(\omega t - \delta)
\]
\[
\delta = \tan^{-1} \frac{\omega L - 1/(\omega C)}{R}
\]

El \textbf{tiempo de retraso} es $\delta/\omega$ segundos.  

%-----------------------------------------------------------------------------------
%-----------------------------------------------------------------------------------
%-----------------------------------------------------------------------------------
%-----------------------------------------------------------------------------------

\section{Parte A: Circuito RC e isoclinas}
\subsection*{a. Ecuación diferencial para la carga \( Q(t) \)}

La ecuación del circuito es:
\[ R \cdot I(t) + \frac{1}{C}Q = E(t) \]
donde \( E(t) = E_0 \) (constante) y \( I(t) = Q'(t) \).

Sustituyendo:
\[ R \frac{dQ}{dt} + \frac{1}{C}Q = E_0 \]

Esta es una ecuación diferencial de primer orden:
\[ \frac{dQ}{dt} + \frac{1}{RC}Q = \frac{E_0}{R} \]

El factor integrante es:
\[ \mu(t) = e^{\int \frac{1}{RC} dt} = e^{\frac{t}{RC}} \]

Multiplicando la ecuación por el factor integrante:
\[ e^{\frac{t}{RC}} \frac{dQ}{dt} + \frac{1}{RC} e^{\frac{t}{RC}} Q = \frac{E_0}{R} e^{\frac{t}{RC}} \]

Reconociendo la derivada del producto:
\[ \frac{d}{dt} \left( Q e^{\frac{t}{RC}} \right) = \frac{E_0}{R} e^{\frac{t}{RC}} \]

Integrando ambos lados:
\[ Q e^{\frac{t}{RC}} = \frac{E_0}{R} \int e^{\frac{t}{RC}} dt = E_0 C e^{\frac{t}{RC}} + K \]

Despejando \( Q(t) \):
\[ Q(t) = E_0 C + K e^{-\frac{t}{RC}} \]

Usando la condición inicial \( Q(0) = 0 \):
\[ 0 = E_0 C + K \Rightarrow K = -E_0 C \]

Por lo tanto:
\[ Q(t) = E_0 C (1 - e^{-\frac{t}{RC}}) \]

La corriente es:
\[ I(t) = \frac{dQ}{dt} = \frac{E_0}{R} e^{-\frac{t}{RC}} \]

\subsection*{b. Verificar que límite \( Q(t) = E_0 C \), límite \( I(t) = 0 \)}

Para la carga:
\[ \lim_{t \to \infty} Q(t) = \lim_{t \to \infty} E_0 C (1 - e^{-\frac{t}{RC}}) = E_0 C (1 - 0) = E_0 C \]

Para la corriente:
\[ \lim_{t \to \infty} I(t) = \lim_{t \to \infty} \frac{E_0}{R} e^{-\frac{t}{RC}} = \frac{E_0}{R} \cdot 0 = 0 \]

\section*{Comparación de los métodos numéricos con la solución analítica}
Para comparar la exactitud de los métodos utilizaremos un ejemplo con los parámetros de: \[E_0 = 110, R = 50, C = 0.05, h = 0.1\] Además usaremos como valor real t = 10 y valor aproximado t = 9,9

\begin{table}[H]
\centering
\begin{tabular}{cc}
\toprule
\textbf{Tiempo (t)} & \textbf{Valor} \\
\midrule
9,9 & 5,399264 \\
10 & 5,395153 \\
\bottomrule
\end{tabular}
\quad
\begin{minipage}{0.5\textwidth}
Conociendo ambos de estos resultados se concluye que el error relativo hacia atrás es de 0.000761
\end{minipage}
\caption{Solución analítica con el valor aproximado y el valor real}
\end{table}

\begin{table}[H]
\centering
\begin{tabular}{cc}
\toprule
\textbf{Tiempo (t)} & \textbf{Valor} \\
\midrule
9,7 & 5,395125 \\
9,8 & 5,399320 \\
9,9 & 5,403347 \\
\bottomrule
\end{tabular}
\quad
\begin{minipage}{0.5\textwidth}
Dando un error relativo hacia adelante de un 0.001517 y un número de condición de 1.993429
\end{minipage}
\caption{Método de Euler}
\end{table}

\begin{table}[H]
\centering
\begin{tabular}{cc}
\toprule
\textbf{Tiempo (t)} & \textbf{Valor} \\
\midrule
9,7 & 5,386299 \\
9,8 & 5,390756 \\
9,9 & 5,395039 \\
\bottomrule
\end{tabular}
\quad
\begin{minipage}{0.5\textwidth}
Dando un error relativo hacia adelante de un 0.000021 y un número de condición de 0.027595
\end{minipage}
\caption{Método Euler Mejorado}
\end{table}

\begin{table}[H]
\centering
\begin{tabular}{cc}
\toprule
\textbf{Tiempo (t)} & \textbf{Valor} \\
\midrule
9,7 & 5,386420 \\
9,8 & 5,390874 \\
9,9 & 5,395153 \\
\bottomrule
\end{tabular}
\quad
\begin{minipage}{0.5\textwidth}
Dando un error relativo hacia adelante de un 0.000761 y un número de condición de 1
\end{minipage}
\caption{Método de Runge - Kutta 4}
\end{table}

\noindent
Con el objetivo de calcular el orden de convergencia de dichos métodos haremos 10 iteraciones de ellos por cada uno de los siguientes valores de h = { 0.5; 0.2; 0.1; 0.05; 0.02; 0.01 } para asi obtener el orden entre cada uno de los tamaños de pasos usando la siguiente fórmula:

\vspace{0.5cm}

\noindent
\textbf{Fórmula del orden de convergencia:} 
\[
p = \frac{\log\left(\frac{\text{Error}_{h_1}}{\text{Error}_{h_2}}\right)}{\log\left(\frac{h_1}{h_2}\right)}
\]

\vspace{0.5cm}

\begin{table}[H]
\centering
\caption{Órdenes de convergencia de los métodos numéricos}
\begin{tabular}{|c|c|c|c|c|c|c|}
\hline
\textbf{Método} & \textbf{0.5→0.2} & \textbf{0.2→0.1} & \textbf{0.1→0.05} & \textbf{0.05→0.02} & \textbf{0.02→0.01} & \textbf{Promedio} \\
\hline
Euler & 1.0595 & 1.0249 & 1.0122 & 1.0055 & 1.0024 & 1.0209 \\
\hline
Euler Mejorado & 2.1025 & 2.0429 & 2.0218 & 2.0099 & 2.0043 & 2.0363 \\
\hline
Runge-Kutta 4 & 4.1104 & 4.0470 & 4.0241 & 4.0109 & 4.0044 & 4.0394 \\
\hline
\end{tabular}
\end{table}

Para visualizar mejor los resultados de la aplicación de estos tres métodos numéricos a la ecuación diferencial dada, se muestran las siguientes gráficas, resultado de la ejecución del primer script de Python del Notebook adjuntado.

\begin{center}
	\includegraphics[width=0.60\textwidth]{solution.png}
	\captionof{figure}{Solución analítica del circuito RC.}
\end{center}

\begin{center}
	\includegraphics[width=0.50\textwidth]{comparison.png}
	\captionof{figure}{Comparación de métodos numéricos (Euler, Euler Mejorado y Runge-Kutta 4) con la solución analítica del circuito RC.}
\end{center}


\subsection*{c. Campo de Isoclinas}
Se adjunta a continuación la representación gráfica del campo de isoclinas, resultado de ejecutar el primer script de Python que contiene el Notebook adjuntado al informe
\begin{center}
	\includegraphics[width=0.50\textwidth]{isoclinas.png}
	\captionof{figure}{Campo de isoclinas del circuito RC.}
\end{center}

\textbf{¿Por qué el comportamiento exponencial en el circuito RC?} 
La forma exponencial surge naturalmente porque la velocidad de carga es proporcional a la carga que falta por acumular, creando este característico crecimiento rápido inicial que luego se frena gradualmente, siendo el término $RC$ la constante de tiempo que determina que tan rápido se carga el capacitor.

\textbf{Comportamiento inicial (t = 0)} 
\begin{itemize}
	\item Al conectar la batería, el capacitor está descargado ($Q = 0$)
	\item Todo el voltaje de la batería $E_0$ cae en la resistencia
	\item La corriente es **máxima**: $I(0) = E_0/R$
	\item La carga comienza a fluir rápidamente al capacitor
\end{itemize}

\textbf{Comportamiento transitorio}
\begin{itemize}
	\item A medida que el capacitor se carga, se opone al paso de más carga
	\item El voltaje en el capacitor aumenta: $V_C = Q/C$
	\item El voltaje disponible para la resistencia disminuye: $V_R = E_0 - Q/C$
	\item La corriente disminuye exponencialmente: $I(t) = \frac{E_0}{R}e^{-t/RC}$
\end{itemize}

\textbf{Comportamiento final $(t \to \infty)$} 
\begin{itemize}
	\item El capacitor alcanza su carga máxima: $Q_{max} = E_0C$
	\item El voltaje del capacitor iguala al de la batería
	\item La corriente se hace cero
	\item El sistema alcanza el equilibrio
\end{itemize}

%-----------------------------------------------------------------------------------
%-----------------------------------------------------------------------------------
%-----------------------------------------------------------------------------------
%-----------------------------------------------------------------------------------

\section{Parte B: Bifurcación}

En esta parte se analiza el comportamiento no lineal de un circuito eléctrico que contiene un diodo, elemento cuya relación corriente-voltaje no es lineal. Este tipo de circuitos puede presentar \textbf{bifurcaciones}, es decir, cambios cualitativos en su dinámica cuando se modifica el voltaje aplicado.

El modelo simplificado que representa este fenómeno se describe mediante la ecuación diferencial:

\[
\frac{dz}{dt} = \mu - z^2,
\]

donde $z(t)$ representa la corriente y $\mu$ es un parámetro de control asociado al voltaje aplicado. Este modelo muestra cómo el comportamiento del circuito cambia según el valor de $\mu$.

\subsection*{a. Determinación de los puntos de equilibrio}

Los puntos de equilibrio se obtienen imponiendo la condición de equilibrio:

\[
\frac{dz}{dt} = 0.
\]

De la ecuación se deduce:
\[
\mu - z^2 = 0,
\]

y por tanto:
\[
z = \pm \sqrt{\mu}.
\]

Esto significa que:
\begin{itemize}
	\item Si $\mu > 0$, existen dos equilibrios reales: $z_1 = +\sqrt{\mu}$ y $z_2 = -\sqrt{\mu}$.
	\item Si $\mu = 0$, ambos colapsan en un único punto $z = 0$.
	\item Si $\mu < 0$, no existen equilibrios reales.
\end{itemize}

\subsection*{b. Estabilidad de los equilibrios}

Para estudiar la estabilidad se linealiza la ecuación alrededor de un equilibrio $z^*$:

\[
\frac{d(\delta z)}{dt} = \left.\frac{d}{dz}(\mu - z^2)\right|_{z = z^*} \delta z.
\]

Como 
\[
\frac{d}{dz}(\mu - z^2) = -2z,
\]
entonces la ecuación linealizada queda:
\[
\frac{d(\delta z)}{dt} = -2z^* \, \delta z.
\]

De aquí:
\begin{itemize}
	\item Si $z^* = +\sqrt{\mu}$, el coeficiente $-2\sqrt{\mu} < 0$ $\Rightarrow$ equilibrio \textbf{estable}.
	\item Si $z^* = -\sqrt{\mu}$, el coeficiente $+2\sqrt{\mu} > 0$ $\Rightarrow$ equilibrio \textbf{inestable}.
\end{itemize}

Por lo tanto, cuando $\mu > 0$, el sistema posee un equilibrio estable y otro inestable.

\subsection*{c. Diagrama de bifurcación}

El diagrama se representa en el plano $(\mu, z)$:

\begin{itemize}
	\item Para $\mu < 0$, no hay equilibrios reales.
	\item En $\mu = 0$, ambos equilibrios coinciden en el origen.
	\item Para $\mu > 0$, aparecen dos ramas: $z = +\sqrt{\mu}$ (estable) y $z = -\sqrt{\mu}$ (inestable).
\end{itemize}

Esto corresponde a una \textbf{bifurcación tipo silla-nodo}, donde dos equilibrios surgen a partir de un valor crítico del parámetro.

Al ejecutar el tercer script en Pyhton del Notebook adjuntado, podemos obtener la representación gráfica del diagrama de bifurcación:

\begin{center}
	\includegraphics[width=0.50\textwidth]{bifurcacion.png}
	\captionof{figure}{Diagrama de bifurcación del modelo $\dot{z} = \mu - z^2$. La rama superior es estable y la inferior inestable.}
\end{center}

\subsection*{d. Interpretación física}

En el contexto del circuito:
\begin{itemize}
	\item $\mu$ representa el \textbf{voltaje aplicado}.
	\item $z(t)$ representa la \textbf{corriente} que atraviesa el diodo.
\end{itemize}

Cuando $\mu < 0$, el voltaje no alcanza el umbral necesario y el sistema no tiene un estado estacionario real: el diodo permanece “apagado”, sin conducción.  
Cuando $\mu > 0$, aparecen dos posibles estados estacionarios:
\begin{itemize}
	\item Uno estable, que corresponde a la corriente positiva (conducción normal del diodo).
	\item Otro inestable, que representaría una corriente negativa físicamente irreal.
\end{itemize}

Así, el cambio de signo de $\mu$ simboliza el paso de un régimen sin corriente a un régimen estable de conducción. Este comportamiento es análogo a la activación de un diodo real al superar su voltaje umbral.

Para concluir, el modelo $\dot{z} = \mu - z^2$ muestra cómo un circuito eléctrico no lineal puede cambiar su comportamiento de forma cualitativa al variar el voltaje aplicado.  
Cuando $\mu$ cruza el valor crítico $0$, el sistema pasa de no tener equilibrios (sin conducción) a tener dos (uno estable y otro inestable), lo que describe perfectamente la aparición de un estado de conducción estable en el circuito con diodo.

Este análisis evidencia cómo las bifurcaciones permiten comprender la transición entre diferentes regímenes dinámicos en sistemas eléctricos no lineales.

%-----------------------------------------------------------------------------------
%-----------------------------------------------------------------------------------
%-----------------------------------------------------------------------------------
%-----------------------------------------------------------------------------------

\section{Parte C: Plano de fase y estabilidad}

\subsection*{Circuito RLC con parámetros numéricos}

El sistema de ecuaciones diferenciales ordinarias dado es el siguiente:
\begin{equation*}
\begin{cases}
\dfrac{dQ}{dt} = I, \\
\dfrac{dI}{dt} = -2I - 5Q,
\end{cases}
\end{equation*}

\subsection*{a. Calcular punto crítico y clasificar sistema según tipo y estabilidad}
\noindent\textbf{Clasificación del sistema según:}
\begin{itemize}
	\item \textbf{Tipo:} Es un sistema de ecuaciones diferenciales ordinarias (derivadas respecto a una sola variable independiente $t$).
	\item \textbf{Orden:} Es un sistema de orden 1 (el mayor orden de derivación que aparece es 1).
	\item \textbf{Linealidad:} Es un sistema lineal (las dos ecuaciones son lineales).
	\item \textbf{Homogeneidad:} Es un sistema homogéneo (ambas ecuaciones son homogéneas).
	\item \textbf{Estabilidad:}Es un sistema estable ya que tiene autovalores complejos con parte real negativa.
\end{itemize}

\noindent\textbf{Representación matricial}

Representación matricial del sistema: 
\[
\begin{bmatrix}
Q' \\
I'
\end{bmatrix}
=
\begin{bmatrix}
0 & 1 \\
-5 & -2
\end{bmatrix}
\cdot
\begin{bmatrix}
Q \\
I
\end{bmatrix}.
\]

Tomándolo como \[X' = AX\] con:
\[
A = \begin{bmatrix} 0 & 1 \\ -5 & -2 \end{bmatrix}, 
\qquad 
X = \begin{bmatrix} Q \\ I \end{bmatrix}.
\]

\noindent\textbf{Resolución del sistema}

Para encontrar los autovalores se calcula
\[
\det(A-\lambda I) = 0.
\]
Obteniendo:
\[
\lambda^2 + 2\lambda + 5 = 0
\]
y por tanto los autovalores son:
\[
\lambda_{1,2} = -1 \pm 2i.
\]

Dado que los autovalores son complejos, tomamos uno de los dos, en este caso \(\lambda_1=-1+2i\), y resolvemos 
\((A-\lambda_1 I)v=0\), obteniendo un autovector complejo:
\[
v=
\begin{bmatrix}
1\\
-1-2i
\end{bmatrix}.
\]

La solución compleja correspondiente es entonces
\[
X(t)=e^{\lambda_1 t}v
= e^{-t}
\begin{bmatrix}
e^{2it}\\
(-1-2i)e^{2it}
\end{bmatrix}.
\]

Usando la identidad \(e^{2it}=\cos(2t)+i\sin(2t)\), se obtiene:
\[
e^{2it}=\cos(2t)+i\sin(2t),
\]
\[
(-1-2i)e^{2it}
= -\cos(2t)-2\sin(2t) 
+ i\,\big(-\sin(2t)+2\cos(2t)\big).
\]

Al separar la solución compleja en sus partes real e imaginaria obtenemos dos soluciones reales linealmente independientes:
\[
X_1(t)=
\begin{bmatrix}
\cos(2t)\\[4pt]
-\cos(2t)-2\sin(2t)
\end{bmatrix}
\]

\[
X_2(t)=
\begin{bmatrix}
\sin(2t)\\[4pt]
-\sin(2t)+2\cos(2t)
\end{bmatrix}.
\]

Los autovectores asociados se hallan resolviendo \((A-\lambda I)v=0\); obteniendo como resultado los siguientes vectores:
\[
v_{\lambda_1} =     
\begin{bmatrix}
    \cos{2t}\\
    -\cos{2t} - 2\sin{2t}
\end{bmatrix}
\]
\[
v_{\lambda_2}=
\begin{bmatrix}
    \sin{2t}\\
    -\sin{2t} + 2\cos{2t}
\end{bmatrix}.
\]

Concluyendo en que la solución general es
\[
X(t) = 
c_1\, v_{\lambda_1} e^{\lambda_1 t}
+
c_2\, v_{\lambda_2} e^{\lambda_2 t}.
\]

\noindent\textbf{Cálculo del punto crítico}

El punto crítico se obtiene imponiendo $Q'=0$ e $I'=0$ en el sistema:
\[
\begin{cases}
I = 0, \\
-2 I - 5 Q = 0.
\end{cases}
\]
De esto se deduce $Q=0$ e $I=0$, luego el punto crítico del sistema es $(0,0)$, siendo este un punto estable.

\subsection*{b. Plano de fase}
A través de la ejecución del último código en Python del Notebook adjuntado a este informe, se puede graficar el plano de fase que corresponde al sistema de ecuaciones diferenciales ordinarias cuyo punto crítico se calculó en la sección anterior.
A continuación la correspondiente gráfica:

\begin{center}
	\includegraphics[width=0.60\textwidth]{fase.png}
	\captionof{figure}{Plano de fase del sistema RLC con punto crítico en el $(0,0)$.}
\end{center}


\par
La figura muestra un \textbf{foco estable} centrado en el origen, ya que todas las trayectorias se aproximan en espiral a $(Q,I)=(0,0)$ para cualquier condición inicial elegida. Esto se debe a que los autovalores de la matriz del sistema son complejos conjugados con parte real negativa, \(\lambda = -1 \pm 2 i\): la parte real \(-1\) provoca un decaimiento exponencial de la amplitud (factor \(e^{-t}\)), mientras que la parte imaginaria \(\pm 2i\) genera oscilaciones con frecuencia angular amortiguada \(\omega_d=2\,\mathrm{rad/s}\) (período \(T = 2\pi/\omega_d = \pi\)).
\par
En términos físicos, las oscilaciones corresponden al intercambio de energía entre el capacitor y el inductor, y la resistencia disipa energía, provocando una reducción progresiva de la amplitud. El sentido de giro es horario: por ejemplo, en el cuadrante superior derecho (\(Q>0, I>0\)) se tiene \(dQ/dt=I>0\) y \(dI/dt=-2I-5Q<0\), lo que desplaza el estado hacia la derecha y hacia abajo, iniciando la rotación en sentido horario.
\par
Las flechas del campo de pendientes indican la dirección y la magnitud instantánea del vector \((dQ/dt, dI/dt)\) en cada punto del plano, mientras que las trayectorias resultantes, trazadas para distintas condiciones iniciales, ilustran cómo el sistema retorna al equilibrio. El origen es el único punto crítico del sistema y actúa como atractor: para todas las condiciones iniciales mostradas, la carga y la corriente tienden a cero a medida que el sistema alcanza el estado de reposo.

%==================================================================================

\label{end}

\end{document}

%===================================================================================